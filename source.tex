\documentclass[11pt,a4paper]{article}

\usepackage[utf8]{inputenc}
\usepackage[T1]{fontenc}
\usepackage{lmodern}
\usepackage{amsmath,amssymb}
\usepackage{graphicx}
\usepackage[left=2cm,top=1.5cm,right=2cm,bottom=1.5cm]{geometry}
\usepackage{hyperref}
\usepackage{enumitem}
\usepackage{xcolor}
\usepackage{array}
\usepackage{titlesec}
\usepackage{parskip}

\pagenumbering{gobble}

\definecolor{sectiontitlecolor}{HTML}{008080}

% Configuração de títulos de seção (REMOVER INDENTAÇÃO)
\titleformat{\section}
  {\color{sectiontitlecolor}\normalfont\Large\bfseries\hangindent=0pt\hangafter=1}
  {}{0pt}{}

\titleformat{\subsection}
  {\color{sectiontitlecolor}\normalfont\large\bfseries\hangindent=0pt\hangafter=1}
  {}{0pt}{}

\renewcommand{\labelitemi}{•}

\begin{document}

\noindent
\begin{tabular}{@{}p{0.5\textwidth}p{0.5\textwidth}@{}}
    \parbox[t]{0.5\textwidth}{\raggedright\Huge \color{sectiontitlecolor}\textbf{Victor Taendy Sousa Emerenciano}} &
    \parbox[t]{0.5\textwidth}{\raggedleft
        São Paulo - SP \\
        (11) 9 5377-5093 \\
        \href{mailto:victor338.flow@gmail.com}{\color{black}victor338.flow@gmail.com} \\
        \href{https://www.linkedin.com/in/victor-taendy/}{\color{black}https://www.linkedin.com/in/victor-taendy/}
    }
\end{tabular}
\vspace{0.3cm}

Cientista de Dados com sólida experiência em desenvolvimento de soluções baseadas em aprendizado de máquina, processamento de imagens e IA generativa. Com foco na entrega de resultados mensuráveis e impacto direto nos processos de negócio. \\
\hrule

\section{Habilidades}
Python, Pandas, Scikit-Learn, TensorFlow, Pytorch | MS SQL Server, Hadoop Hive, AWS Athena, SageMaker Studio | OpenCV, NLTK, SpaCy, XGBoost, LightGBM | Git, GitHub | Análise de dados \\
\hrule

\section{Experiência}

\subsection{2023 – 2025 Cientista de Dados Pleno – Itaú Unibanco}
Monitoramento de Ambientes com Imagens
\begin{itemize}[label=•]
	\item Implementação de modelos de detecção de entidades (YOLO) para identificar presença humana em ambientes específicos, utilizando soluções da Ultralytics e OpenCV.
	\item Desenvolvimento de modelo de classificação com PyTorch para atuar como filtro preliminar, reduzindo a carga computacional dos modelos de detecção.
\end{itemize}

Priorização de Operações para Quality Assurance
\begin{itemize}[label=•]
	\item Desenvolvimento de modelo de classificação utilizando Naive Bayes para melhorar a seleção de operações críticas para análise, superando a abordagem de amostragem aleatória em 0,5 ponto percentual na métrica Average Precision.
	\item Criação de visualizações para comunicar efetivamente os ganhos do modelo em termos de recall e precisão para stakeholders não técnicos.
\end{itemize}

Extração de informações não padronizadas com IA generativa
\begin{itemize}[label=•]
	\item Desenvolvi uma solução baseada em IA generativa para extrair informações críticas de contratos societários, caracterizados por linguagem natural e estrutura não padronizada.
	\item Superando as limitações dos métodos tradicionais de detecção de entidades, resultando em maior agilidade na atualização de cenários empresariais e eficiência operacional.
\end{itemize}

\subsection{2021 – 2022 Cientista de Dados Junior – Itaú Unibanco}
Classificação de alto desbalanceio para operações suspeitas
\begin{itemize}[label=•]
	\item Desenvolvi um modelo de ranqueamento (XGBoost “learn-to-rank”) para priorizar operações suspeitas em um cenário com alto desbalanceamento de classes, elevando a métrica de 5\% com técnicas anteriores baseadas em Random Forest para 11\% com o modelo atual.
	\item Essa melhoria permitiu uma triagem mais eficiente das transações, mantendo o custo com fraudes igual mesmo com um aumento substancial de tentativas de ataque.
\end{itemize}

\vspace{0.3cm}
\hrule

\section{Formação}
\begin{tabular}{@{}ll@{}}
    2014 – 2020: & Bacharelado em Sistemas de Informação \\
                          & Escola de Artes Ciências e Humanidades da Universidade de São Paulo
\end{tabular}

\end{document}